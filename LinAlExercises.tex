% Document built on template from Jennifer Pan, http://jenpan.com/jen_pan/PsetLatexTemplate.tex

\documentclass[10pt,letter]{article}
	% basic article document class
	% use percent signs to make comments to yourself -- they will not show up.

\usepackage{amsmath}
\usepackage{amssymb}
	% packages that allow mathematical formatting

\usepackage{graphicx}
	% package that allows you to include graphics

\usepackage{setspace}
	% package that allows you to change spacing

\onehalfspacing
	% text become 1.5 spaced

\usepackage{fullpage}
	% package that specifies normal margins
	

\begin{document}
	% line of code telling latex that your document is beginning


\title{Linear Algebra Exercises}

\author{Rayhan Momin}

\date{August 2, 2017}
	% Note: when you omit this command, the current dateis automatically included
 
\maketitle 
	% tells latex to follow your header (e.g., title, author) commands.

\section*{Exercise 1}

\begin{flushleft}
Let $x$ be a given $n \times 1$ vector and consider the problem
\end{flushleft}

\[v(x) =  \max_{y,u} \left\{ - y'P y - u' Q u \right\}\]

\begin{flushleft}
subject to the linear constraint

\end{flushleft}
\[y = Ax + Bu\]

\begin{flushleft}
Here: 
\end{flushleft}

\begin{itemize}
  \item $P$ is an $n \times n$ matrix and $Q$ is an $m \times m$ matrix
  \item $A$ is an $n \times n$ matrix and $B$ is an $n \times m$ matrix
  \item both $P$ and $Q$ are symmetric and positive semidefinite
\end{itemize}

\begin{flushleft}
One way to solve this problem is to form the Lagrangian
\end{flushleft}

\[\mathcal L = - y' P y - u' Q u + \lambda' \left[A x + B u - y\right]\]

\begin{flushleft}
where $\lambda$ is an $n \times 1$ vector of Lagrange multipliers
\newline
Show that these conditions imply that
\end{flushleft}

\begin{enumerate}
\item $\lambda = -2Py$
\item The optimizing choice of $u$ satisfies $u = - (Q + B' P B)^{-1} B' P A x$
\item The function $v$ satisfies $v(x) = - x' \tilde P x$ where $\tilde P = A' P A - A'P B (Q + B'P B)^{-1} B' P A$
\end{enumerate}

\section*{My Solution to Exercise 1}

\paragraph{A)} Answer to Problem 1(A) here.

\subparagraph{i)} Answer to Problem 1(A)(i) here.

\subparagraph{ii)} Answer to Problem 1(A)(ii) here.

\subparagraph{iii)} Answer to Problem 1(A)(iii) here.

\end{document}
	% line of code telling latex that your document is ending. If you leave this out, you'll get an error
